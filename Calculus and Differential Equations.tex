\documentclass{article}
\usepackage{template}

\usepackage{chngcntr} % Reset counter within sections

\counterwithin*{equation}{section}
\counterwithin*{equation}{subsection}

\pagestyle{fancy}
\setlength\headheight{24pt}

\lhead{\className}
\rhead{\leftmark}
\cfoot{\thepage}

\newcommand{\uniTitle}{Queensland University of Technology}
\newcommand{\className}{Calculus and Differential Equations}
\newcommand{\classTime}{2021, Semester 2}
\newcommand{\classInstructorName}{Vivien Challis}
\newcommand{\authorName}{Tarang Janawalkar}
\newcommand{\authorStudentNumber}{n11032201}
\newcommand{\classCode}{MXB105}

\usepackage[
    type={CC},
    modifier={by-nc-sa},
    version={4.0},
    imagewidth={5em},
]{doclicense}

\date{}

\begin{document}

\begin{titlepage}
    \vspace*{\fill}
    \begin{center}
        \LARGE
        \textbf{\className}
        \texorpdfstring{\\}{ }
        \uniTitle
        \texorpdfstring{\\}{ }
        \texorpdfstring{\vspace{0.3in}}{ }
        \normalsize\textit{\classInstructorName}
        \texorpdfstring{\\}{ }
        \classTime
    \end{center}
    \begin{center}
        \textbf{\authorName}
    \end{center}
    \vspace*{\fill}
    \doclicenseThis
    \thispagestyle{empty}
\end{titlepage}
\newpage

\tableofcontents
\newpage

\section{Integration}
\subsection{Derivatives}
Let $a\in\mathbb{R}$ be a constant.
\begin{table}[H]
    \renewcommand*{\arraystretch}{1.5}
    \centering
    \begin{tabular}{>{$}c<{$} | >{$}c<{$}}
        \toprule
        f & \dv{f}{x} \\
        \midrule
        a & 0 \\
        x^a & a x^{a-1} \\
        a^x & \ln{\left( a \right)} a^x \\
        \log_a{x}, \: a\in \mathbb{R}\backslash\left\{ 0 \right\} & \frac{1}{a\ln{x}} \\
        \bottomrule
    \end{tabular}
    \begin{tabular}{>{$}c<{$} | >{$}c<{$}}
        \toprule
        f & \dv{f}{x} \\
        \midrule
        x & 1 \\
        \sqrt{x} & \frac{1}{2\sqrt{x}} \\
        \e^x & \e^x \\
        \ln{x} & \frac{1}{x} \\
        \bottomrule
    \end{tabular}
    \caption{Elementary Function Derivatives}
\end{table}
\begin{table}[H]
    \renewcommand*{\arraystretch}{1.5}
    \centering
    \begin{tabular}{>{$}c<{$} | >{$}c<{$}}
        \toprule
        f & \dv{f}{x} \\
        \midrule
            \sin{\left( ax \right)} &  a\cos{\left( ax \right)} \\
            \cos{\left( ax \right)} & -a\sin{\left( ax \right)} \\
            \tan{\left( ax \right)} &  a\sec^2{\left( ax \right)} \\
            \cot{\left( ax \right)} & -a\csc^2{\left( ax \right)} \\
            \sec{\left( ax \right)} &  a\sec{\left( ax \right)}\tan{\left( ax \right)} \\
            \csc{\left( ax \right)} & -a\csc{\left( ax \right)}\cot{\left( ax \right)} \\
        \bottomrule
    \end{tabular}
    \begin{tabular}{>{$}c<{$} | >{$}c<{$}}
        \toprule
            f & \dv{f}{x} \\
        \midrule
            \sinh{\left( ax \right)} &  a\cosh{\left( ax \right)} \\
            \cosh{\left( ax \right)} &  a\sinh{\left( ax \right)} \\
            \tanh{\left( ax \right)} &  a\sech^2{\left( ax \right)} \\
            \coth{\left( ax \right)} & -a\csch^2{\left( ax \right)} \\
            \sech{\left( ax \right)} & -a\sech{\left( ax \right)}\tan{\left( ax \right)} \\
            \csch{\left( ax \right)} & -a\csch{\left( ax \right)}\cot{\left( ax \right)} \\
        \bottomrule
    \end{tabular}
    \caption{Trigonometric Function Derivatives}
\end{table}
\begin{table}[H]
    \renewcommand*{\arraystretch}{1.5}
    \centering
    \begin{tabular}{>{$}c<{$} | >{$}c<{$}}
        \toprule
            f & \dv{f}{x} \\
        \midrule
            \arcsin{\left( ax \right)} &  \frac{a}{\sqrt{1-a^2x^2}} \\
            \arccos{\left( ax \right)} & -\frac{a}{\sqrt{1-a^2x^2}} \\
            \arctan{\left( ax \right)} &  \frac{a}{1+a^2x^2} \\
            \arccot{\left( ax \right)} & -\frac{a}{1+a^2x^2} \\
            \arcsec{\left( ax \right)} &  \frac{1}{x\sqrt{a^2x^2 - 1}} \\
            \arccsc{\left( ax \right)} & -\frac{1}{x\sqrt{a^2x^2 - 1}} \\
        \bottomrule
    \end{tabular}
    \begin{tabular}{>{$}c<{$} | >{$}c<{$}}
        \toprule
            f & \dv{f}{x} \\
        \midrule
            \arcsinh{\left( ax \right)} &  \frac{a}{\sqrt{1+a^2x^2}} \\
            \arccosh{\left( ax \right)} &  \frac{a}{\sqrt{1-a^2x^2}} \\
            \arctanh{\left( ax \right)} &  \frac{a}{1-a^2x^2} \\
            \arccoth{\left( ax \right)} &  \frac{a}{1-a^2x^2} \\
            \arcsech{\left( ax \right)} & -\frac{1}{a\left( 1+ax \right)\sqrt{\frac{1-ax}{1+ax}}} \\
            \arccsch{\left( ax \right)} & -\frac{1}{ax^2\sqrt{1+\frac{1}{a^2x^2}}} \\
        \bottomrule
    \end{tabular}
    \caption{Inverse Trigonometric Function Derivatives}
\end{table}
\subsection{Partial Fractions}
\begin{definition}[Partial Fraction Decomposition]
    \textbf{Partial fraction decomposition} is a \linebreak method where a rational function $\frac{P(x)}{Q(x)}$ is rewritten as a sum of fraction.
\end{definition}
\begin{table}[H]
    \renewcommand*{\arraystretch}{1.5}
    \centering
    \begin{tabular}{c | c}
        \toprule
            Factor in denominator & Term in partial fraction decomposition \\
        \midrule
            $ax+b$ & $\frac{A}{ax+b}$ \\
            $\left(ax+b\right)^k$ & $\frac{A_1}{ax+b} + \frac{A_2}{\left( ax+b \right)^2} + \cdots + \frac{A_k}{\left( ax+b \right)^k}, \: k \in \mathbb{N}$ \\
            $ax^2+bx+c$ & $\frac{A}{ax^2+bx+c}$ \\
            $\left(ax^2+bx+c\right)^k$ & $\frac{A_1x+B_1}{ax^2+bx+c} + \frac{A_2}{\left( ax+b \right)^2} + \cdots + \frac{A_k}{\left( ax+b \right)^k}, \: k \in \mathbb{N}$ \\
        \bottomrule
    \end{tabular}
    \caption{Partial Fraction Forms}
\end{table}
\subsection{Integration by Parts}
\begin{theorem}
\begin{equation*}
    \int u \dd{v} = uv - \int v \dd{u}
\end{equation*}
\end{theorem}
\begin{proof}
    \begin{align*}
        \dv{}{x}\left( u(x)v(x) \right) &= \dv{u(x)}{x}v(x) + u(x)\dv{v(x)}{x} \\
        u(x)\dv{v(x)}{x} &= \dv{}{x}\left( u(x)v(x) \right) - \dv{u(x)}{x}v(x) \\
        \int u(x)\dv{v(x)}{x} \dd{x} &= \int \dv{}{x}\left( u(x)v(x) \right) \dd{x} - \int \dv{u(x)}{x}v(x) \dd{x} \\
        \int u(x)\dd{v(x)} &= u(x)v(x) - \int v(x) \dd{u(x)}
    \end{align*}
\end{proof}
\subsection{Integration by Substitution}
\begin{theorem}
    \begin{equation*}
        \int f\left(g\left( x \right)\right)\dv{g(x)}{x} \dd{x} = \int f(u) \dd{u}, \: \text{where } u = g(x)
    \end{equation*}
\end{theorem}
\subsection{Trigonometric Identities}
\begin{equation*}
    \sin^2{\left( \theta \right)} + \cos^2{\left( \theta \right)} = 1
\end{equation*}
\begin{equation*}
    \tan^2{\left( \theta \right)} + 1 = \sec^2{\left( \theta \right)}
\end{equation*}
\begin{equation*}
    1 + \cot^2{\left( \theta \right)} = \csc^2{\left( \theta \right)}
\end{equation*}

\subsection{Trigonometric Substitutions}
\begin{table}[H]
    \centering
    \begin{tabular}{c | c | }
        \toprule
        
        \midrule
        
        \bottomrule
    \end{tabular}
    % \caption{}
    % \label{}
\end{table}
\newpage
\section{Limits, Continuity and Differentiability}

\newpage
\section{Definite Integrals}
\newpage
\section{Taylor and Maclaurin Series}
\newpage
\section{Multivariable Functions}
\newpage
\section{Double and Triple Integrals}
\newpage
\section{Vector-Valued Functions}
\newpage
\section{First-Order Differential Equations}
\newpage
\section{Second-Order Differential Equations}
\newpage

\end{document}
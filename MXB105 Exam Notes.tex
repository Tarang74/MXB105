%!TEX TS-program = xelatex
%!TEX options = -aux-directory=Debug -shell-escape -file-line-error -interaction=nonstopmode -halt-on-error -synctex=1 "%DOC%"
\documentclass{article}
\input{LaTeX-Submodule/template.tex}

% Additional packages and macros
\usepackage{changepage} % Modify page width
\usepackage{multicol} % Use multiple columns
\usepackage{titlesec} % Modify section heading styles

%% A4 page
\geometry{
	a4paper,
	margin = 10mm
}

%% Hide horizontal rule
\renewcommand{\headrulewidth}{0pt}

%% Hide page numbers
\pagenumbering{gobble}

%% Multi-columns setup
\setlength\columnsep{4pt}

%% Paragraph setup
\setlength\parindent{0pt}
\setlength\parskip{0pt}

%% Customise section heading styles
\titleformat*\section{\raggedright\bfseries}

\begin{document}
% Modify spacing
\titlespacing*\section{0pt}{1ex}{1ex}
%
\setlength\abovecaptionskip{-5pt}
\setlength\textfloatsep{0pt}
%
\setlength\abovedisplayskip{1pt}
\setlength\belowdisplayskip{1pt}
%
\begin{multicols}{3}
    \section*{Derivative Rules}
    \begin{table}[H]
        \centering
        \begin{tabular}{c c}
            \toprule
            \(f\left( x \right)\)                                        & \(f'\left( x \right)\)                                        \\
            \midrule
            \(u\left( x \right)v\left( x \right)\)                       & \(u'v + uv'\)                                                 \\
            \(\displaystyle\frac{u\left( x \right)}{v\left( x \right)}\) & \(\displaystyle\frac{u'v - uv'}{v^2}\)                        \\
            \(u\bigl( v\left( x \right) \bigr)\)                         & \(u'\bigl( v\left( x \right) \bigr)v'\left( x \right)\)       \\
            \midrule
            \(x^{n}\)                                                    & \(nx^{n - 1}\)                                                \\
            \(\ln{\bigl( u\left(x\right) \bigr)}\)                       & \(\displaystyle\frac{u'\left( x \right)}{u\left( x \right)}\) \\
            \midrule
            \(\sin{\left( ax \right)}\)                                  & \(a\cos{\left( ax \right)}\)                                  \\
            \(\cos{\left( ax \right)}\)                                  & \(-a\sin{\left( ax \right)}\)                                 \\
            \(\tan{\left( ax \right)}\)                                  & \(a\sec^2{\left( ax \right)}\)                                \\
            \(\cot{\left( ax \right)}\)                                  & \(-a\csc^2{\left( ax \right)}\)                               \\
            \(\sec{\left( ax \right)}\)                                  & \(a\sec{\left( ax \right)}\tan{\left( ax \right)}\)           \\
            \(\csc{\left( ax \right)}\)                                  & \(-a\csc{\left( ax \right)}\cot{\left( ax \right)}\)          \\
            \bottomrule
        \end{tabular}
    \end{table}
    \section*{Trigonometric Identities}
    \begin{align*}
        1                        & = \sin^2{\left( x \right)} + \cos^2{\left( x \right)} \\
        \sin{\left( 2x \right)}  & = 2\sin{\left( x \right)}\cos{\left( x \right)}       \\
        \cos{\left( 2x \right)}  & = \cos^2{\left( x \right)} - \sin^2{\left( x \right)} \\
        \sin^2{\left( x \right)} & = \frac{1-\cos{\left( 2x \right)}}{2}                 \\
        \cos^2{\left( x \right)} & = \frac{1+\cos{\left( 2x \right)}}{2}
    \end{align*}
    \section*{Partial Fraction Decomposition}
    Given the LHS in the denominator, substitute the RHS\@.
    \begin{align*}
        \left(ax+b\right)^k & \to \frac{A_1}{ax+b} + \cdots + \frac{A_k}{\left( ax+b \right)^k}
    \end{align*}
    \begin{align*}
        \left(ax^2+bx+c\right)^k     & \to                                                   \\
        \frac{A_1x+B_1}{ax^2+bx+c} + & \cdots + \frac{A_k x+B_k}{\left( ax^2+bx+c \right)^k}
    \end{align*}
    \section*{Integration Techniques}
    \begin{align*}
        \int u \odif{v}                                                        & = uv - \int v \odif{u}            \\
        \int f\bigl(g\left( x \right)\bigr)\odv{g\left( x \right)}{x} \odif{x} & = \int f\left( u \right) \odif{u}
    \end{align*}
    where \(u = g\left( x \right)\).
    \section*{Trigonometric Substitutions}
    \begin{table}[H]
        \centering
        \begin{tabular}{c c}
            \toprule
            \textbf{Form}  & \textbf{Substitution}                        \\
            \midrule
            \(a^2-b^2x^2\) & \(x=\frac{a}{b}\sin{\left( \theta \right)}\) \\
            \(a^2+b^2x^2\) & \(x=\frac{a}{b}\tan{\left( \theta \right)}\) \\
            \(b^2x^2-a^2\) & \(x=\frac{a}{b}\sec{\left( \theta \right)}\) \\
            \bottomrule
        \end{tabular}
    \end{table}
    \section*{L'Hôpital's Rule}
    If \(\displaystyle \lim_{x\to x_0}f\left( x \right)=\lim_{x\to x_0}g\left( x \right)=0\) or \(\pm\infty\), then
    \(\displaystyle \lim_{x\to x_0}\frac{f\left( x \right)}{g\left( x \right)} = \lim_{x\to x_0}\frac{f'\left( x \right)}{g'\left( x \right)}\).
    \section*{Continuity}
    \(f\left( x \right)\) continuous at \(c\) iff \(\displaystyle \lim_{x\to c} f\left( x \right) = f\left( c \right)\).

    \(f\left( x \right)\) is continuous on \(I:\left( a,\:b \right)\) if it is continuous
    for all \(x\in I\).

    \(f\left( x \right)\) is continuous on \(I:\left[ a,\:b \right]\) if it is continuous
    for all \(x\in I\), but only right continuous at \(a\) and left continuous at \(b\).
    \section*{Intermediate Value Theorem}
    If \(f\left( x \right)\) is continuous on \(I:\left[ a, \: b \right]\) and \(f\left( a \right) \leq c \leq f\left( b \right)\), then \(\exists x\in I:f\left( x \right)=c\).
    \section*{Differentiability}
    \(f\left( x \right)\) is differentiable at \(x=x_0\) iff
    \begin{equation*}
        f'\left( x_0 \right) = \lim_{x\to x_0} \frac{f\left( x \right)-f\left( x_0 \right)}{x-x_0}
    \end{equation*}
    exists. This defines the derivative
    \begin{equation*}
        f'\left( x_0 \right) = \lim_{h\to 0} \frac{f\left( x_0+h \right)-f\left( x_0 \right)}{h}.
    \end{equation*}
    Differentiability implies continuity.
    \section*{Mean Value Theorem}
    If \(f\left( x \right)\) is continuous and differentiable on \(I:\left[ a,\:b \right]\), then
    \begin{equation*}
        \exists c\in I:f'\left( c \right)=\frac{f\left( b \right)-f\left( a \right)}{b-a}.
    \end{equation*}
    \section*{Definite Integrals}
    \begin{equation*}
        A = \int_a^b f\left( x \right) \odif{x}
    \end{equation*}
    \section*{Fundamental Theorem of Calculus}
    \begin{align*}
        \int_a^b \odv{F\left( x \right)}{x} \odif{x}                 & = F\left( b \right) - F\left( a \right) \\
        \odv*{\left( \int_a^x f\left( t \right) \odif{t} \right)}{x} & = f\left( x \right)
    \end{align*}
    \section*{Taylor Polynomials}
    \begin{equation*}
        f\left( x \right) \approx p_n\left( x \right) = \sum_{k=0}^n \frac{f^{\left( k \right)}\left( x_0 \right)}{k!} \left( x-x_0 \right)^k
    \end{equation*}
    \section*{Taylor Series}
    \begin{equation*}
        f\left( x \right) = \sum_{n=0}^{\infty} \frac{f^{\left( n \right)}\left( x_0 \right)}{n!}\left( x-x_0 \right)^n
    \end{equation*}
    \textbf{Maclaurin Series:} \(x_0 = 0\).
    \section*{Common Maclaurin Series}
    \begin{table}[H]
        \centering
        \begin{tabular}{c c c}
            \toprule
            \textbf{Function}           & \textbf{Series Term}                                          & \textbf{Conv.}             \\
            \midrule
            \(e^{x}\)                   & \(\frac{x^n}{n!}\)                                            & all \(x\)                  \\
            \(\sin{\left( x \right)}\)  & \(\left( -1 \right)^n \frac{x^{2n+1}}{\left( 2n+1 \right)!}\) & all \(x\)                  \\
            \(\cos{\left( x \right)}\)  & \(\left( -1 \right)^n \frac{x^{2n}}{\left( 2n \right)!}\)     & all \(x\)                  \\
            \(\frac{1}{1-x}\)           & \(x^n\)                                                       & \(\left( -1,\: 1 \right)\) \\
            \(\frac{1}{1+x^2}\)         & \(\left( -1 \right)^n x^{2n}\)                                & \(\left( -1,\: 1 \right)\) \\
            \(\ln{\left( 1+x \right)}\) & \(\left( -1 \right)^{n+1} \frac{x^n}{n}\)                     & \(\left( -1,\: 1 \right]\) \\
            \bottomrule
        \end{tabular}
    \end{table}
    \textbf{Power Series:} \(\sum_{n=0}^{\infty} c_n\left( x-x_0 \right)^n\)
    \section*{Series Tests}
    For a series of the form \(\displaystyle\sum_{i=i_0}^\infty a_i\):
    \section*{Alternating Series Test}
    Given \(a_i = \left( -1 \right)^i b_i\) and \(b_i>0\).

    If \(b_{i+1}\leqslant b_i\) \& \(\lim_{i\to\infty}b_i=0\), then
    convergent, else inconclusive. \section*{Ratio Test} Given \(\rho =
    \lim_{i\to\infty}\abs{\frac{a_{i+1}}{a_i}}\):
    \begin{align*}
        \rho < 1 & : \text{convergent}   \\
        \rho > 1 & : \text{divergent}    \\
        \rho = 1 & : \text{inconclusive}
    \end{align*}
    \section*{Multivariable Functions}
    \begin{equation*}
        f:\mathbb{R}^n\to\mathbb{R}
    \end{equation*}
    \section*{Level Curves}
    \begin{equation*}
        L_c\left( f \right) = \bigl\{ \left( x,\: y \right) : f\left(x,\: y\right) = c\bigr\}
    \end{equation*}
    If \(f\left( x,\: y \right) \to L\) as \(\left( x,\: y \right) \to \left( x_0,\: y_0 \right)\), then
    \(\displaystyle \lim_{\left( x,\: y \right) \to \left( x_0,\: y_0 \right)} = L\) along any smooth
    curve. The limit does not exist if \(L\) changes along different smooth curves.

    \textbf{Partial Derivatives:} w.r.t one variable, others held constant.

    \textbf{Gradient:} \(\symbf{\nabla} = \abracket{\pdif*{x_1},\: \pdif*{x_2},\: \dots,\: \pdif*{x_n}}\)
    \section*{Multivariable Chain Rule}
    For \(f=f\bigl(\symbf{x}\left( t_1,\: \ldots,\: t_n \right)\bigr)\) with
    \(\symbf{x}=
    \begin{bmatrix}
        x_1 & \cdots & x_m
    \end{bmatrix}
    \)
    \begin{equation*}
        \pdv{f}{t_i} = \symbf{\nabla}f \cdot \pdv{\symbf{x}}{t_i}.
    \end{equation*}
    \section*{Directional Derivatives}
    \begin{align*}
        \symbf{\nabla}_{\symbf{u}}f
         & = \symbf{\nabla}f \cdot \hat{\symbf{u}}
    \end{align*}
    where \(\hat{\symbf{u}}\) is a unit vector and the slope is given by \(\norm{\symbf{\nabla}_{\symbf{u}}f}\).
    If \(\symbf{\nabla}_{\symbf{u}}f=0\), \(\symbf{u}\) is tangent to the level curve at \(\symbf{x}_0\).
    \begin{align*}
        \max_{\norm{\symbf{u}} = 1} \symbf{\nabla}_{\symbf{u}}f = \symbf{\nabla}f
    \end{align*}
    If \(\symbf{\nabla}f\neq 0\), \(\symbf{\nabla}f\) is a normal vector to the level curve at \(\symbf{x}_0\).
    \section*{Critical Points}
    \(\left( x_0,\: y_0 \right)\) is a critical point if \(\symbf{\nabla}f\left( x_0,\: y_0 \right) = \symbf{0}\)
    or if \(\symbf{\nabla}f\left( x_0,\: y_0 \right)\) is undefined.
    \section*{Classification of Critical Points}
    \begin{equation*}
        D = f_{xx}f_{yy} - \left( f_{xy} \right)^2
    \end{equation*}
    \begin{description}
        \item[\(D > 0\) and \(f_{xx} < 0\):] local maxima
        \item[\(D > 0\) and \(f_{xx} > 0\):] local minima
        \item[\(D < 0\):] saddle point
        \item[\(D = 0\):] inconclusive
    \end{description}
    \section*{Double Integrals}
    The volume of the solid
    enclosed between the surface \(z=f\left( x,\: y \right)\) and the region \(\Omega\) is
    defined by
    \begin{equation*}
        V = \iint\limits_{\Omega} f\left( x,\: y \right) \odif{A}.
    \end{equation*}
    If \(\Omega\) is a region bounded by \(a \leq x \leq b\) and \(c \leq y \leq d\), then
    \begin{align*}
        \iint\limits_{\Omega} f\left( x,\: y \right) \odif{A} & = \int_c^d\int_a^b f\left( x,\: y \right) \odif{x} \odif{y} \\
                                                              & = \int_a^b\int_c^d f\left( x,\: y \right) \odif{y} \odif{x}
    \end{align*}
    \section*{Type I Regions}
    \begin{equation*}
        \iint\limits_{\Omega} f\left( x,\: y \right) \odif{A} = \int_a^b\int_{g_1}^{g_2} f\left( x,\: y \right) \odif{y} \odif{x}
    \end{equation*}
    \textbf{Bounded left \& right by:}
    \begin{equation*}
        x=a \text{ and } x=b
    \end{equation*}
    \textbf{Bounded below \& above by:}
    \begin{equation*}
        y=g_1\left( x \right) \text{ and } y=g_2\left( x \right)
    \end{equation*}
    where \(g_1\left( x \right) \leq g_2\left( x \right)\) for \(a \leq x \leq b\):
    \section*{Type II Regions}
    \begin{equation*}
        \iint\limits_{\Omega} f\left( x,\: y \right) \odif{A} = \int_c^d\int_{h_1}^{h_2} f\left( x,\: y \right) \odif{x} \odif{y}
    \end{equation*}
    \textbf{Bounded left \& right by:}
    \begin{equation*}
        x=h_1\left( y \right) \text{ and } x=h_2\left( y \right)
    \end{equation*}
    \textbf{Bounded below \& above by:}
    \begin{equation*}
        y=c \text{ and } y=d
    \end{equation*}
    where \(h_1\left( y \right) \leq h_2\left( y \right)\) for
    \(c \leq y \leq d\). To integrate, solve the inner integrals first.
    \section*{Vector Valued Functions}
    \begin{equation*}
        \symbf{r}:\mathbb{R} \to \mathbb{R}^n
    \end{equation*}
    The \textbf{domain} of \(\symbf{r}\left( t \right)\) is the
    intersection of the domains of its components.
    The \textbf{orientation} of \(\symbf{r}\left( t \right)\) is the
    direction of motion along the curve as the value of the parameter
    increases. \textbf{Limits}, \textbf{derivatives} and
    \textbf{integrals} are all \textit{component-wise}. Each component
    has its \underline{own} constant of integration.
    \section*{Parametric Lines}
    \begin{equation*}
        \symbf{l}\left( t \right) = \symbf{P}_0 + t\symbf{v}
    \end{equation*}
    where \(\symbf{l}\left( t \right)\) passes through \(\symbf{P}_0\), and is parallel to \(\symbf{v}\).
    \section*{Tangent Lines}
    If \(\symbf{r}\left( t \right)\) is differentiable at
    \(t_0\) and \(\symbf{r'}\left( t_0 \right)\ne\symbf{0}\)
    \begin{equation*}
        \symbf{l}\left( t \right) = \symbf{r}\left( t_0 \right)+t\symbf{r'}\left( t_0 \right).
    \end{equation*}
    \section*{Curves of Intersection}
    Choose one of the variables as the parameter, and express the remaining variables in terms of that parameter.
    \section*{Arc Length}
    \begin{equation*}
        S = \int_a^b \norm{\symbf{r'}\left( t \right)} \odif{t}
    \end{equation*}
    \section*{Ordinary Differential Equations}
    \textbf{Order:} highest derivative in DE\@.

    \textbf{Autonomous DE:} does not depend explicitly on the independent variable.
    \section*{Qualitative Analysis}
    \begin{equation*}
        \odv{y}{t} = f\left( y \right)
    \end{equation*}
    A fixed point is the value of \(y\) for which \(f\left( y \right) = 0\).
    \section*{Stability of Fixed Points}
    Given a positive/negative perturbation from a fixed point, that point is

    \textbf{Stable:} if both tend toward FP

    \textbf{Unstable:} if both tend away from FP

    \textbf{Semi-Stable:} if one tends toward FP, and another tends away from FP
    \section*{Directly Integrable ODEs}
    For \(\odv{y}{x} = f\left( x \right)\):
    \begin{equation*}
        y\left( x \right) = \int f\left( x \right) \odif{x}.
    \end{equation*}
    \section*{Separable ODEs}
    For \(\odv{y}{x} = p\left( x \right) q\left( y \right)\):
    \begin{equation*}
        \int \frac{1}{q\left( y \right)} \odv{y}{x} \odif{x} = \int p\left( x \right) \odif{x}.
    \end{equation*}
    \section*{Linear ODEs}
    For \(\odv{y}{x} + p\left( x \right)y = q\left( x \right)\), use the \textit{integrating factor}:
    \(I\left( x \right) = e^{\int p\left( x \right) \odif{x}}\), so that
    \begin{equation*}
        y\left( x \right) = \frac{1}{I\left( x \right)} \int I\left( x \right) q\left( x \right) \odif{x}.
    \end{equation*}
    \section*{Exact ODEs}
    \(P\left( x,\: y \right) + Q\left( x,\: y \right)\odv{y}{x} = 0\)
    has the solution
    \(\Psi\left( x,\: y \right) = c\)
    iff it is exact, namely, when
    \(P_y = Q_x\),
    where \(P = \Psi_x\) and \(Q = \Psi_y\). Then
    \begin{gather*}
        \Psi\left( x,\: y \right) = \int P\left( x,\: y \right) \odif{x} + f\left( y \right) \\
        \Psi\left( x,\: y \right) = \int Q\left( x,\: y \right) \odif{y} + g\left( x \right)
    \end{gather*}
    and \(f\left( y \right)\) and \(g\left( x \right)\) can be determined by solving these equations simultaneously.
    \section*{Second-Order ODEs}
    \begin{equation*}
        a_2\left( x \right)y'' + a_1\left( x \right)y' + a_0\left( x \right)y = F\left( x \right)
    \end{equation*}
    \section*{Initial Values}
    \begin{align*}
        y\left( x_0 \right) = y_0 \quad y'\left( x_0 \right) = y_1
    \end{align*}
    \section*{Boundary Values}
    \begin{align*}
        y\left( x_0 \right) = y_0 \quad y\left( x_1 \right) = y_1
    \end{align*}
    \section*{Reduction of Order}
    \begin{equation*}
        y_2\left( x \right) = v\left(x\right) y_1\left( x \right)
    \end{equation*}
    \(v\left( x \right)\) can be determined by substituting \(y_2\) into the ODE, using \(w\left( x \right) = v'\left( x \right)\).
    \section*{General Solution}
    \begin{equation*}
        y\left( x \right) = y_H\left( x \right) + y_P\left( x \right)
    \end{equation*}
    \section*{Homogeneous Solution}
    \begin{equation*}
        y_H\left( x \right) = e^{\lambda x}
    \end{equation*}
    \section*{Real Distinct Roots}
    \begin{equation*}
        y_H\left( x \right) = c_1e^{\lambda_1 x} + c_2e^{\lambda_2 x}
    \end{equation*}
    \section*{Real Repeated Roots}
    \begin{equation*}
        y_H\left( x \right) = c_1e^{\lambda x} + c_2 te^{\lambda x}
    \end{equation*}
    \section*{Complex Conjugate Roots}
    Given \(\lambda = \alpha \pm \beta i\):
    \begin{equation*}
        y_H\left( x \right) = e^{\alpha x}\bigl( c_1\cos{\left( \beta x \right)} + c_2 \sin{\left( \beta x \right)} \bigr)
    \end{equation*}
    \section*{Particular Solution}
    \emph{See table below.}
    Substitute \(y_P\) into the nonhomogeneous ODE, and solve the undetermined coefficients.
    \section*{Spring and Mass Systems}
    \begin{equation*}
        m y'' + \gamma y' + k y = f\left( t \right)
    \end{equation*}
    \textbf{Newton's Law:} \(F = m y''\)

    \textbf{Spring force:} \(F_s = -k y\)

    \textbf{Damping force:} \(F_d = -\gamma y'\)
    \begin{align*}
        m      & : \text{mass}    & k                 & : \text{spring constant} \\
        \gamma & : \text{damping} & f\left( t \right) & : \text{external force}
    \end{align*}
    \section*{Electrical Circuits}
    The sum of voltages around a loop equals 0.
    \begin{align*}
        v\left( t \right) - iR - L \odv{i}{t} - \frac{q}{C} & = 0                 \\
        L \odv[2]{q}{t} + R\odv{q}{t} + \frac{1}{C}q        & = v\left( t \right)
    \end{align*}
    where \(\displaystyle i = \odv{q}{t}\).
    \section*{Voltage drop across various elements:}
    \begin{align*}
        v_R & = iR           \\
        v_C & = \frac{q}{C}  \\
        v_L & = L \odv{i}{t}
    \end{align*}
    \begin{align*}
        R & : \text{resistance} & C                 & : \text{capacitance}    \\
        L & : \text{inductance} & v\left( t \right) & : \text{voltage supply}
    \end{align*}
\end{multicols}
\begin{table}[H]
    \centering
    \begin{tabular}{c c}
        \toprule
        \(F\left( x \right)\)                                                  & \(y_P\left( x \right)\)                                                   \\
        \midrule
        a constant                                                             & \(A\)                                                                     \\
        a polynomial of degree \(n\)                                           & \(\sum_{i = 0}^n A_i x^i\)                                                \\
        \(e^{kx}\)                                                             & \(A e^{kx}\)                                                              \\
        \(\cos{\left( \omega x \right)}\) or \(\sin{\left( \omega x \right)}\) & \(A_0 \cos{\left( \omega x \right)} + A_1 \sin{\left( \omega x \right)}\) \\
        a combination of the above                                             & a combination of the above                                                \\
        linearly dependent to \(y_H\left( x \right)\)                          & multiply \(y_P\left( x \right)\) by \(x\) until linearly independent      \\
        \bottomrule
    \end{tabular}
\end{table}
\end{document}

